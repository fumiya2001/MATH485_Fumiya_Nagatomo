\documentclass[12pt]{article}
\usepackage[margin=1in]{geometry}
\usepackage{amsmath}
\usepackage{booktabs}
\usepackage{array}
\usepackage{amssymb} % for \square
\usepackage{xcolor}
\usepackage{hyperref}


\newcolumntype{C}[1]{>{\centering\arraybackslash}p{#1}}
\newcommand{\blank}{\(\square\)} 


\setlength{\parindent}{0pt}
\setlength{\parskip}{4pt}

\begin{document}

\begin{center}
{\LARGE\bfseries Assignment \#1: Probability Models}
\end{center}

\section*{General Instructions}

\begin{itemize}
  \item There are two problems. Carefully read all notes and questions.
  \item Answers should be clearly typed as in a technical report.
  \item Type out the whole solutions and answers, submit a PDF report on Canvas (or other associated material).
  \item Scores for rubrics are shown in [ ] within each question.
\end{itemize}

\section*{Problem 1: Naive Bayes for Customer Churn Using Observed Data [50 pts]}

\subsection*{Background}

You are a data scientist working for a subscription-based company. Your goal is to predict whether a customer will churn ($Y=1$) or not churn ($Y=0$) using behavioral data.

You will build two Naive Bayes models, each using a different feature set. For each model, you will:
\begin{enumerate}
  \item Estimate probabilities from data
  \item Apply Laplace smoothing
  \item Compute posterior churn probabilities
  \item Compare model conclusions
\end{enumerate}

\subsection*{Given (common to both models)}

Let $Y$ represent the event churn. From historical data, the overall churn rate is:
\begin{itemize}
  \item $P(Y=1) = 0.3$
  \item $P(Y=0) = 0.7$
\end{itemize}

\subsection*{Model A --- Support-related features}

\subsection*{Feature definitions}

\begin{itemize}
  \item $(X_1 = 1)$: Customer opened many support tickets last month
  \item $(X_2 = 1)$: Customer spent a lot of time on help / FAQ pages
\end{itemize}

\subsection*{Training dataset (historical sample)}

\begin{tabular}{|c|c|c|c|}
\hline
Churn (Y) & $X_1=1$ & $X_1=0$ & Total \\
\hline
$Y=1$ & 24 & 6 & 30 \\
\hline
$Y=0$ & 14 & 56 & 70 \\
\hline
\end{tabular}

\vspace{4mm}

\begin{tabular}{|c|c|c|c|}
\hline
Churn (Y) & $X_2=1$ & $X_2=0$ & Total \\
\hline
$Y=1$ & 25 & 5 & 30 \\
\hline
$Y=0$ & 13 & 57 & 70 \\
\hline
\end{tabular}

\vspace{2mm}
(Note: each feature table is estimated independently, as in Naive Bayes.)

For an observed customer with $X_1=1, X_2=1$, answer the following questions:

\subsection*{Question A1 --- Estimate probabilities}

Using \textbf{Laplace smoothing}, estimate:
\begin{itemize}
  \item [5 pts] $P(X_1=1 \mid Y=1)$, $P(X_1=1 \mid Y=0)$
  \item [5 pts] $P(X_2=1 \mid Y=1)$, $P(X_2=1 \mid Y=0)$
\end{itemize}

\subsection*{Answer for A1}
$P(X_1=1 \mid Y=1) = \frac{24 + 1}{30 + 2} = \frac{25}{32} 
\approx 0.781$

$P(X_1=1 \mid Y=0) = \frac{14 + 1}{70 + 2} = \frac{15}{72} 
\approx 0.208$

$P(X_2=1 \mid Y=1) = \frac{25 + 1}{30 + 2} = \frac{26}{32} 
\approx 0.812$

$P(X_2=1 \mid Y=0) = \frac{13 + 1}{70 + 2} = \frac{14}{72} 
\approx 0.194$
\subsection*{Question A2 --- Posterior probabilities}

Using your smoothed estimates and Naive Bayes, compute the
\textbf{normalized posterior probabilities}:
\begin{itemize}
  \item [4 pts] $P(Y=1 \mid X_1=1, X_2=1)$
  \item [4 pts] $P(Y=0 \mid X_1=1, X_2=1)$
\end{itemize}

\subsection*{Answer for A2}

\[
\begin{aligned}
P(Y=1 \mid X_1=1, X_2=1)
&= P(Y=1)P(X_1=1 \mid Y=1)P(X_2=1 \mid Y=1) \\
&= \frac{3}{10} \cdot \frac{25}{32} \cdot \frac{26}{32}
\approx 0.1904
\end{aligned}
\]

\[
\begin{aligned}
P(Y=0 \mid X_1=1, X_2=1)
&= P(Y=0)P(X_1=1 \mid Y=0)P(X_2=1 \mid Y=0) \\
&= \frac{7}{10} \cdot \frac{15}{72} \cdot \frac{14}{72}
\approx 0.0283
\end{aligned}
\]

Then, compute normalized posterior probabilities based on the result above

$P(Y=1 \mid X_1=1, X_2=1) = \frac{0.1904}{0.1904 + 0.0283} \approx 0.8703$

$P(Y=0 \mid X_1=1, X_2=1) = \frac{0.0283}{0.1904 + 0.0283} \approx 0.1296$

\subsection*{Question A3 --- Prediction}

Based on Model A:
\begin{itemize}
  \item [1 pts] Does the model predict churn or no churn?
  \item [1 pts] What is the predicted probability?
\end{itemize}

\subsection*{Answer for A3}
\begin{itemize}
  \item The model predicts churn (Y = 1), since
  $P(Y=1 \mid X_1=1, X_2=1) > P(Y=0 \mid X_1=1, X_2=1)$.
  \item The predicted probability of churn is
  $P(Y=1 \mid X_1=1, X_2=1) \approx 0.87$.
\end{itemize}


\subsection*{Model B --- Pricing and competition features}

\subsection*{Feature definitions}

\begin{itemize}
  \item $(Z_1 = 1)$: Customer changed pricing plan last month
  \item $(Z_2 = 1)$: Customer received a competitor's marketing email
\end{itemize}

\subsection*{Training dataset (historical sample)}

\begin{tabular}{|c|c|c|c|}
\hline
Churn (Y) & $Z_1=1$ & $Z_1=0$ & Total \\
\hline
$Y=1$ & 18 & 12 & 30 \\
\hline
$Y=0$ & 21 & 49 & 70 \\
\hline
\end{tabular}

\vspace{4mm}

\begin{tabular}{|c|c|c|c|}
\hline
Churn (Y) & $Z_2=1$ & $Z_2=0$ & Total \\
\hline
$Y=1$ & 15 & 15 & 30 \\
\hline
$Y=0$ & 28 & 42 & 70 \\
\hline
\end{tabular}

For an observed customer with $Z_1=1, Z_2=1$, answer the following questions:

\subsection*{Question B1 --- Estimate probabilities}

Using \textbf{Laplace smoothing}, estimate:
\begin{itemize}
  \item [5 pts] $P(Z_1=1 \mid Y=1)$, $P(Z_1=1 \mid Y=0)$
  \item [5 pts] $P(Z_2=1 \mid Y=1)$, $P(Z_2=1 \mid Y=0)$
\end{itemize}

\subsection*{Answer for B1}
$P(Z_1=1 \mid Y=1) = \frac{18 + 1}{30 + 2} = \frac{19}{32} 
\approx 0.593$

$P(Z_1=1 \mid Y=0) = \frac{21 + 1}{70 + 2} = \frac{22}{72} 
\approx 0.305$

$P(Z_2=1 \mid Y=1) = \frac{15 + 1}{30 + 2} = \frac{16}{32} 
= 0.5$

$P(Z_2=1 \mid Y=0) = \frac{28 + 1}{70 + 2} = \frac{29}{72} 
\approx 0.402$
\subsection*{Question B2 --- Posterior probabilities}

Compute the normalized posterior probabilities:
\begin{itemize}
  \item [4 pts] $P(Y=1 \mid Z_1=1, Z_2=1)$
  \item [4 pts] $P(Y=0 \mid Z_1=1, Z_2=1)$
\end{itemize}

\subsection*{Answer for B2}

\[
\begin{aligned}
P(Y=1 \mid Z_1=1, Z_2=1)
&= P(Y=1)P(Z_1=1 \mid Y=1)P(Z_2=1 \mid Y=1) \\
&= \frac{3}{10} \cdot \frac{19}{32} \cdot \frac{16}{32}
\approx 0.0890
\end{aligned}
\]

\[
\begin{aligned}
P(Y=0 \mid Z_1=1, Z_2=1)
&= P(Y=0)P(Z_1=1 \mid Y=0)P(Z_2=1 \mid Y=0) \\
&= \frac{7}{10} \cdot \frac{22}{72} \cdot \frac{29}{72}
\approx 0.0861
\end{aligned}
\]

Then, compute normalized posterior probabilities based on the result above

$P(Y=1 \mid Z_1=1, Z_2=1) = \frac{0.0890}{0.0890 + 0.0861} \approx 0.5083$

$P(Y=0 \mid Z_1=1, Z_2=1) = \frac{0.0861}{0.0890 + 0.0861} \approx 0.4916$

\subsection*{Question B3 --- Prediction}

Based on Model B:
\begin{itemize}
  \item [1 pts] Does the model predict churn or no churn?
  \item [1 pts] What is the predicted probability?
\end{itemize}

\subsection*{Answer for B3}
\begin{itemize}
  \item The model predicts churn (Y = 1), since
  $P(Y=1 \mid Z_1=1, Z_2=1) > P(Y=0 \mid Z_1=1, Z_2=1)$.
  \item The predicted probability of churn is
  $P(Y=1 \mid Z_1=1, Z_2=1) \approx 0.5083$.
\end{itemize}

\subsection*{Model comparison and interpretation}

Compare the predictions from Model A and Model B:
\begin{itemize}
  \item [5 pts] Why are the predicted churn probabilities so different?
  \item [5 pts] Which model's prediction is more trustworthy? Why?
\end{itemize}

\subsection*{Answer}
\begin{itemize}
  \item The reason why the predicted churn probabilities are so different is because 
  the two models use the different feature with different strengths of association with churn.
  \item Model A is more trustworthy because the posterior probability is much higher and further
  from the prior. On the other hand, model B's prediction is close to 0.5, which is less useful to predict.
\end{itemize}
\section*{Problem 2 [50 pts]}

Complete the Hidden Markov Model (HMM) Forward/Backtracking table using the following transition and emission probabilities.

\subsection*{Transition}

\begin{tabular}{|c|c|c|c|}
\hline
From $\rightarrow$ To & Sunny & Cloudy & Rainy \\
\hline
Sunny & 0.33 & 0.67 & 0.00 \\
Cloudy & 0.33 & 0.00 & 0.67 \\
Rainy & 0.33 & 0.33 & 0.33 \\
\hline
\end{tabular}

\subsection*{Emission}

\begin{tabular}{|c|c|c|}
\hline
Weather $\rightarrow$ Behavior & Walk & Umbrella \\
\hline
Sunny & $4/4 = 1.0$ & $0/3 = 0.0$ \\
Cloudy & $2/3 \approx 0.67$ & $1/3 \approx 0.33$ \\
Rainy & $1/3 \approx 0.33$ & $2/3 \approx 0.67$ \\
\hline
\end{tabular}

\vspace{4mm}
You can manually compute (show the steps), code (submit through GitHub), or use a spreadsheet (submit the spreadsheet) to work out the solutions.

\subsection*{Answer for Problem2}

I summarized what I did in the file `HMM.ipynb`':
\url{https://github.com/fumiya2001/Math485/blob/main/HW01/HMM.ipynb}
by using a table.

\begin{center}
\begin{tabular}{|c|c|c|c|c|c|}
\hline
Day & Observation & ? $->$ & Sunny & Cloudy & Rainy \\
\hline
 &  & $V0(?)$ & 0.33 & 0.33 & 0.33 \\
\hline
1 & Walk & $P(W \mid ?)$ & 1.00 &  0.67 & 0.33  \\
\hline
& &$V1(?) = V0(?)P(W \mid ?)$ & 0.33 & \textcolor{red}{0.22} & 0.11\\
\hline
&  & & & &\\
\hline
2 & & $V1(S)P(? \mid S)$ & 0.11 & 0.22 & 0.00 \\
\hline
& & $V1(C)P(? \mid C)$ & 0.07 & 0.00 & \textcolor{red}{0.15} \\
\hline
& & $V1(R)P(? \mid R)$ & 0.04 & 0.04  & 0.04\\
\hline
& Umbrella & $P(U \mid ?)$ & 0 & 0.33 & 0.67 \\
\hline
& & $V2(?) = \max(?)P(U \mid ?)$ & 0 & 0.07 & \textcolor{red}{0.10} \\
\hline
&  & & & &\\
\hline
3 & & $V2(S)P(? \mid S)$ & 0.00 & 0.00 & 0.00 \\
\hline
& & $V2(C)P(? \mid C)$ & 0.02 & 0.00 & 0.05 \\
\hline
& & $V2(R)P(? \mid R)$ & \textcolor{red}{0.03} & 0.03  & 0.03\\
\hline
& Walk & $P(W \mid ?)$ & 1.00 & 0.67 & 0.33 \\
\hline
& & $V3(?) = \max(?)P(U \mid ?)$ & \textcolor{red}{0.03} & 0.02 & 0.02 \\
\hline
\end{tabular}
\end{center}


\end{document}
